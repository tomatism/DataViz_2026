\documentclass[12pt,letterpaper]{article}
\usepackage{graphicx,textcomp}
\usepackage{natbib}
\usepackage{setspace}
\usepackage{fullpage}
\usepackage{color}
\usepackage[reqno]{amsmath}
\usepackage{amsthm}
\usepackage{fancyvrb}
\usepackage{amssymb,enumerate}
\usepackage[all]{xy}
\usepackage{endnotes}
\usepackage{lscape}
\newtheorem{com}{Comment}
\usepackage{float}
\usepackage{hyperref}
\newtheorem{lem} {Lemma}
\newtheorem{prop}{Proposition}
\newtheorem{thm}{Theorem}
\newtheorem{defn}{Definition}
\newtheorem{cor}{Corollary}
\newtheorem{obs}{Observation}
\usepackage[compact]{titlesec}
\usepackage{dcolumn}
\usepackage{tikz}
\usetikzlibrary{arrows}
\usepackage{multirow}
\usepackage{xcolor}
\newcolumntype{.}{D{.}{.}{-1}}
\newcolumntype{d}[1]{D{.}{.}{#1}}
\definecolor{light-gray}{gray}{0.65}
\usepackage{url}
\usepackage{listings}
\usepackage{color}

\definecolor{codegreen}{rgb}{0,0.6,0}
\definecolor{codegray}{rgb}{0.5,0.5,0.5}
\definecolor{codepurple}{rgb}{0.58,0,0.82}
\definecolor{backcolour}{rgb}{0.95,0.95,0.92}

\lstdefinestyle{mystyle}{
	backgroundcolor=\color{backcolour},   
	commentstyle=\color{codegreen},
	keywordstyle=\color{magenta},
	numberstyle=\tiny\color{codegray},
	stringstyle=\color{codepurple},
	basicstyle=\footnotesize,
	breakatwhitespace=false,         
	breaklines=true,                 
	captionpos=b,                    
	keepspaces=true,                 
	numbers=left,                    
	numbersep=5pt,                  
	showspaces=false,                
	showstringspaces=false,
	showtabs=false,                  
	tabsize=2
}
\lstset{style=mystyle}
\newcommand{\Sref}[1]{Section~\ref{#1}}
\newtheorem{hyp}{Hypothesis}

\title{Problem Set 1}
\date{Due: January 28, 2026}
\author{Data Visualisation for Social Scientists}

\begin{document}
	\maketitle
	
	\section*{Instructions}
	\begin{itemize}
	\item Please show your work! You may lose points by simply writing in the answer. If the problem requires you to execute commands in \texttt{R}, please include the code you used to get your answers. Please also include the \texttt{.R} file that contains your code. If you are not sure if work needs to be shown for a particular problem, please ask.
\item Your homework should be submitted electronically on GitHub.
\item This problem set is due before 23:59 on Wednesday January 28, 2026. No late assignments will be accepted.
	\end{itemize}
	
	\vspace{1cm}
	\section*{Roll Call Votes in the European Parliament}

\subsection*{Data Manipulation}
First, you need to \href{https://personal.lse.ac.uk/hix/HixNouryRolandEPdata.HTM}{download data} from the first six elected European Parliaments on each MEP and how they voted in each recorded roll-call vote.

\vspace{.25cm}

\begin{enumerate}
\item Load these datasets into your global environment:
\begin{itemize}
	\item \texttt{mep\_info\_26Jul11.xls} (MEP characteristics, EP1–EP5)
	\item \texttt{rcv\_ep1.txt} (EP1 roll-call votes)
\end{itemize}

\item Briefly describe (2–3 sentences each) the unit of analysis and key variables in each of these two datasets.

\item The \texttt{rcv\_ep1} data are in a wide format, with V1, V2, …, Vn as separate vote columns.
\begin{itemize}
	\item Identify which columns are ID/metadata (\textit{MEPID, MEPNAME, MS, NP, EPG}) and which columns are vote decisions ($V_1$…$V_n$). Tidy the voting data such that each row/observation is a single vote for a single MEP.
	\item Create a summary table of counts of decision categories (e.g. Yes/No/Abstain/Present but did not vote/Absent) across all votes.
\end{itemize}
\item  Construct a new dataset that combines MEP-level information with their vote decisions from EP1 in long format (from part 3). Check for missingness.
\item Compute, for each EP group in EP1:
\begin{itemize}
	\item The mean rate of Yes votes (Yes over Yes+No+Abstain) across all roll calls.
	\item The mean abstention rate.
	\item The mean vote preferences along the two contested dimensions (NOM-D1 and NOM-D2).
\end{itemize}
\end{enumerate}

\subsection*{Data Visualization}

\begin{enumerate}
	\item Plot the distribution of the first NOMINATE dimension by EP group, and explain any trends you see.
	\item Make a scatterplot of \textit{nomdim1} (x-axis) and \textit{nomdim2} (y-axis), with one point per MEP and color by EP group.
	\item Produce a boxplot of the proportion voting \textit{Yes} by EP group to visualize cohesion.
	\item Display the proportion voting \textit{Yes} per year by national party using a bar plot.
	\item For each EP group, calculate the average \textit{Yes} share per year and plot a line graph. 
\end{enumerate}


\end{document}
